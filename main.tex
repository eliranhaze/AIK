\documentclass[12pt]{article}

\usepackage[fulloldstylenums]{kpfonts}
%\usepackage[colorlinks]{hyperref}
%\usepackage[margin=1in]{geometry}
\usepackage[onehalfspacing]{setspace}
\usepackage{enumitem}
\setlist[description]{leftmargin=\parindent,labelindent=\parindent} % for indented description environment
\setlist[enumerate]{label*=(\arabic*)} % set the style of numbering in enum env

\usepackage[en-US]{datetime2}
\DTMlangsetup*{showdayofmonth=false}

% my commands for this paper
\usepackage{commands}

\usepackage{xcolor}
\definecolor{citec}{HTML}{3a5082} % colors from: https://learnui.design/tools/data-color-picker.html
\definecolor{linkc}{HTML}{7c568f}
\definecolor{urlc}{HTML}{b55c87}
\usepackage[
    colorlinks,
    citecolor=citec,
    linkcolor=linkc,
    urlcolor=urlc,
]{hyperref}

%\emergencystretch=1em % to avoid overflowing lines in bibliography
\usepackage[
    backend=biber,
    style=chicago-authordate,
    doi=false,
    isbn=false,
    url=false,
]{biblatex}
\addbibresource{references.bib}

\AtEveryBibitem{\clearfield{note}}

\title{Assertion, Implicature, and Iterated Knowledge}
\author{Eliran Haziza}
%\date{}

\begin{document}

\maketitle

Abstract:
Two important contributions of this papers: adds to the growing literature on sub-assertoric speech and its epistemic norms, and adds another linguistic consideration for the KK principle (adding to Dorst).

I argue that whatever norm governs a speech act, a condition on that norm is that the implicatures of the act must be known.

\section{Introduction}

. . .

\section{Knowledge and Implicature}

. . .
What I want to argue, really, is that the cancellability conditions for implicature are equal to its permissibility conditions.

An implicature is cancelled when its truth is denied, when belief is denied, and when knowledge is denied, and when justification is denied (``But I have no reason to think this'')

\section{Knowledge Implicatures}

In this section I shall argue that a speaker's assertion that \textit{p} gives rise to a \textit{knowledge implicature}, i.e., an implicature that the speaker knows \textit{p}. This will be important for the argument later on that if both assertion and implicature are subject to a knowledge norm, then the KK thesis is true.

\subsection{How implicatures are generated}

One way in which implicatures are generated, on a broadly Gricean view, is by taking the speaker to be observing the conversational maxims.\footnote{See e.g. \textcites[p.~44f]{birner_introduction_2013}[33]{huang_pragmatics_2015}[158]{huang_implicature_2017}[103-4]{levinson_pragmatics_1983}.} For each conversational maxim there is a kind of implicature associated with observing that maxim. For instance, for the maxim of Quantity (\textit{Be as informative as required}), there are scalar implicatures:

\begin{enumerate}[resume]
  \item\label{ex:quant}
  I ate most of the pizza.\\
  $\leadsto$ I didn't eat all of the pizza.
\end{enumerate}
%
Given the assumption that the speaker is observing Quantity, an assertion of \ref{ex:quant} generates the implicature that the speaker didn't eat all of the pizza, since if the speaker did and was as informative as required, she would have said `all' and not `most'. The maxim of Relation (\textit{Be relevant}) can generate implicatures of relevance:

\begin{enumerate}[resume]
  \item\label{ex:rel}
    A: I am out of gas.\\
    B: There's a station down the street.\\
    $\leadsto$ You can get gas there.
\end{enumerate}
%
In \ref{ex:rel}, B's answer conveys an implicature (perhaps among others) that A can get gas at the station. The implicature is generated by taking B to be saying something relevant to A's question. A third case involves the maxim of Manner (\textit{Be perspicuous}).

\begin{enumerate}[resume]
  \item\label{ex:mann}
    Ann drove home and drank a bottle of wine.\\
    $\leadsto$ Ann drank a bottle of wine after driving home.
\end{enumerate}
%
An assertion of \ref{ex:mann}, on the assumption that the speaker is observing Manner, conveys the implicature that the event described by the first conjunct happened before that described by the second.

Finally, the maxim of Quality (\textit{Do not say what you belief is false or for which you lack evidence}) generates \textit{Quality implicatures}:

\begin{enumerate}[resume]
  \item\label{ex:qual}
    A: Where is Bob?\\
    B: He's in his office.\\
    $\leadsto$ I believe that Bob is in his office.
\end{enumerate}
%
If assertion has a knowledge norm, then speakers are not only required to believe and have evidence for what they assert, as Quality enjoins; they are also required to know that. As \textcite{benton_gricean_2016} argues, the knowledge norm of assertion does a similar job to Quality, and perhaps supersedes it. Knowledge implicatures arise in the same way that Quality implicatures do: by taking the speaker to be observing the norm of assertion. On a knowledge account of assertion, Quality implicatures are implicatures of knowledge. By asserting \textit{p}, the speaker implicates that she knows \textit{p}.

At this point the reader might worry: Aren't implicatures supposed to be cancellable? If they are, there can't be any belief or knowledge implicatures since those are clearly not cancellable:

\begin{enumerate}[resume]
  \item
    \# Bob is in his office, but I don't believe that.
  \item
    \# Bob is in his office, but I don't know that.
\end{enumerate}
%
I shall argue, however, that there are good reasons to think that not all implicatures are cancellable. Understanding when implicatures are cancellable will give us a better understanding of the norms of conversation. I shall argue that an implicature is cancellable only if the resulting utterance complies with some conversational norm. Many implicatures are cancellable because cancelling them still complies with the maxim of Quality. The fact that knowledge implicatures are not cancellable is thus further evidence for the knowledge norm of assertion.

\subsection{Cancellability}

Cancellability is often taken to be a necessary feature of implicatures: if an implication of an assertion cannot be cancelled, it is not a conversational implicature. This is taken to be one of the essential differences between conversational implicature and other sorts of implications such as presuppositions and logical entailments. But this conventional wisdom is only partly true.

It is important first to distinguish between \textit{explicit} and \textit{implicit} cancellation. It is true that all implicatures are implicitly cancellable. More precisely, if an utterance of \textit{p} in one context implicates \textit{q}, then there is another context in which an utterance of \textit{p} does not implicate \textit{q}. This is \textit{contextual} cancellation. Explicit cancellation, on the other hand, is cancellation by means of explicit negation. If an utterance of \textit{p} implicates \textit{q} in context \textit{c}, and "\textit{p} but not \textit{q}" or "\textit{p} but I don't mean to imply \textit{q}" can be felicitously uttered in \textit{c}, then we say that \textit{q} is explicitly cancellable.\footnote{See \textcite{geurts_quantity_2010}. (Also note that some implicatures can be cancelled by meta-linguistic negation. Huang mentions this and probably Geurts too.}

All implicatures are---probably, though I will not defend this---contextually cancellable. But not all implicatures are explicitly cancellable. \textcite{weiner_are_2006} gives the following example:

\begin{enumerate}[resume]
  \item\label{ex:train}
    Hannah and Sarah are in a crowded train, Hannah is sprawled across two seats and Sarah is standing next to her, sandwiched in a crowd of strangers.\\
    Sarah: I am wondering whether it would be physically possible for you to make room for another person to sit down.\\
    $\leadsto$ You should make room for me to sit down.
\end{enumerate}
%
Weiner argues that Sarah cannot explicitly cancel the implicature. It's not that the resulting utterance would be infelicitous; it's just that it would seem to double down on the implicature:

\begin{enumerate}[resume]
  \item\label{ex:train2}
    Sarah: I am wondering whether it would be physically possible for you to make room for another person to sit down. But I'm not implying that you should make room.\\
    $\leadsto$ You should make room.
\end{enumerate}
%
One cannot help but hear Sarah as being sarcastic. However, one does not have to look too far. The following is adapted from \textcite{akerman_infelicitous_2015}:

\begin{enumerate}[resume]
  \item\label{ex:yesno}
    Ann: Are you happy at your new job?\\
    Bob: Yes and no.
\end{enumerate}
%
In \ref{ex:yesno}, Bob implies that he is happy in some respects but not in others. But a felicitous cancellation of this implicature does not seem possible:

\begin{enumerate}[resume]
  \item\label{ex:yesno2}
    Ann: Are you happy at your new job?\\
    Bob: Yes and no. \# But I don't mean to imply that I'm happy in some respects and not in others.
\end{enumerate}
%
The \textit{New Yorker} reports the following exchange, taking place in meeting to consider promotions for several staffers. The CEO, Brown, questions the whole idea of tiered titles.

\begin{enumerate}[resume]
  \item\label{ex:ibm}
    Woodside: Last week, we were very close to promoting eight or nine people, and now we’re going to take everyone's titles away?\\
    Brown: Is there a way to have a more sensible system that wasn't invented for I.B.M.?
\end{enumerate}

Brown's utterance implicates that the system in question was invented for I.B.M or similar companies. But the implicature does not seem cancellable:

\begin{enumerate}[resume]
  \item\label{ex:ibm2}
    Brown: Is there a way to have a more sensible system that wasn't invented for I.B.M.? \# But I don't mean to imply that the system was invented for I.B.M.
\end{enumerate}

Consider also:

\begin{enumerate}[resume]
  \item\label{ex:wine-spec}
    Ann: Shall we try this wine?\\
    Bob: I'm not going to drink a bottle of wine that has been sitting in the sun for five and a half hours.
\end{enumerate}
%
Bob's answer in \ref{ex:wine-spec} implicates that the bottle under discussion has been sitting in the sun for five and a half hours. Clearly, this is only an implicature and not something that Bob asserts. However, it is not felicitously cancellable:

\begin{enumerate}[resume]
  \item\label{ex:wine-spec2}
    Ann: Shall we try this wine?\\
    Bob: I'm not going to drink a bottle of wine that has been sitting in the sun for five and a half hours. \# But this wine has not been sitting in the sun for five and a half hours.
\end{enumerate}
%
Why is Bob's implicature not cancellable in \ref{ex:wine-spec2}, while a similar implicature is cancellable, as seen in the following exchange:

\begin{enumerate}[resume]
  \item\label{ex:wine-gen}
    Ann: Shall we try this wine?\\
    Bob: I don't drink cheap wine, but this wine isn't cheap.
\end{enumerate}
%
In \ref{ex:wine-gen}, Bob's answer may be seen as overly informative or just generally annoying, but it is not infelicitous. The implicature that the wine in question is cheap is felicitously cancelled. In \ref{ex:wine-spec2} Bob's implicature cannot be cancelled felicitously. What accounts for the difference? I propose that the difference is that in \ref{ex:wine-gen} there's a way to interpret Bob: he is being annoying or arrogant, but he is still answering the question, at least implicitly. It is possible to interpret Bob's reply as a positive reply, that he is after all willing to try the present wine. In \ref{ex:wine-spec2}, however, Bob's cancellation leaves no way of making sense of him. One cannot understand why Bob is saying what he is saying if the implicature is not true. I suggest, then, the following necessary condition on explicit cancellability:

\begin{description}
    \item[Cancellability] An implicature is explicitly cancellable in context \textit{c} only if asserting not-\textit{q} ...
\end{description}

\printbibliography

\end{document}

Say the following:
* On a broadly Gricean view, implicatures are generated by ... or by ...
. Examples: ...
* Work that assumes Quality Implicatures. Given one detailed example. E.g., in the influential analysis of scalar implicatures by Fox (2007), it is assumed that the speaker knows p or q. ...

How implicatures are generated:
- Huang 2015: either observing or flouting the maxims. (Possibly Grice himself says this.)
- According to the Generative Assumption (See the 2008 version of SEP/Implicature and corresponding footnote with refs)
- Levinson 1983, around 103: implicatures are generated because speakers are assumed to adhere to the maxims (whether they seem to or not). Explicitly in 104: "These inferences come about in at least two distinct ways..."
- Huang, Levinson, and Grice himself should suffice.
- Also look at: Davis (handbook)[V], Huang (handbook)[V], Blome-Tillman (Compass)[V], Geurts (Intro)[V], Meibauer (Encyc)[V], Birner [V]. Findings:
-- Birner accepts Quality implicatures (pp. 65-66). Writes that there are 4 ways of relating to maxims: observing, violating, flouting, and opting out. The first 3 correspond to 3 ways of generating implicatures. (p. 42f, and p. 74 for summary)
-- Huang (Handbook, p. 158): implicatures are generated in two distinct ways: from observing the maxims (e.g. "soup is warm" +> "soup is not hot"), or from openly foulting them (e.g. "the british foreign office is in washington" +> "uk follows usa too closely".
-- Huang 2014, p. 33: same thing, more detailed.
-- Davis: nothing.
-- Geurts: ψ is a conversational implicature licensed by the speaker’s utterance of ϕ if (i) ψ is not conventionally associated with ϕ and (ii) the hearer has good reasons for believing that, assuming the speaker intends to be cooperative, he wouldn’t have uttered ϕ unless ψ was the case.
-- Geurts: derivation of implicatures:
i. It starts with a speaker S performing a particular kind of action: S utters a sentence ϕ.
ii. In view of the current discourse goals, S’s utterance wouldn’t have been in accordance with the Cooperation Principle unless ψ was the case.
iii. Hence, ψ must be the case.
-- Geurts: The implicature, i.e. ψ, will generally be of the form “S believes that ...”,
-- Meibauer 2006: "Conversational implicatures come about by the exploitation (apparent flouting) or observation of the cooperative principle (CP) and a set of maxims" Then goes on to list examples of each, including Belief Implicatures.
-- BT: Nothing.

I should write: Grice denied that there are belief and knowledge implicatures. But this has not gained favor. For instance, all of . . . accept that belief or knowledge implicatures are generated via Quality.
(I should say that an assertion invites the inference/supposition that the speaker knows)

Who accepts Quality implicatures:
- Cremers et al
- Ciardelli et al
- Levinson 1983
- Huang 2015; 2017
- Birner 2013
- Davis?
- Schwarz 2016 (calls the process of generating implicatures "the standard recipe")
- Geurts 2010 (e.g. p. 41)
- Buccola \& Haida 2019
- Meyer 2013 also assumes belief/knowledge implicatures (p. 11f, 20)
- Mendia 2016; ms (mentions "Quality Implicatures")
- Gazdar 1979
- Fox 2007
- Franke 2011 (S&P)
- Stokke 2013 (M&L) argues that hearers typically infer "speaker believes p" from "speaker said p" in order to explain Protagonist Projection. Does not commit himself to the claim that it's an implicature,  but it really should be.
- Guerts 2009, section 6.1 (Again uses the implicature that B(pvq) in the context of scalar implicatures. Many theorists seem to assume this in order to derive some explanation.) (perhaps this is what is called the "Standard Recipe"?)
- Ninan 2014
- Paillé and Schwarz 2019, also explains the B(QvP) stuff.
- Meibauer 2014? Mentions belief implicatures, but can't tell if endorses them (p. 77-8)
- Ward et al 2003
- Kay 1992
- Soames 1982 (appendix, discussing Gazdar)
- Zufferey et al p. 133 (discussing Gazdar) perhaps accept Quality implicatures. See also p. 136 discussion of Fox 2007.
- From other doc:
-- Gazdar 1979 argues that Quality gives rise to knowledge implicatures, and argues that not all implicatures are cancellable.
 [In fact (see Soames's 1982 A55) his is an early account of Quality as knowledge.]
-- So does Levinson 1983; 2000, arguing that quality implicatures are not cancellable.
-- Meibauer 2006 claims that there are belief implicatures.
-- DeRose and Grandy 1999 argue that there are quality implicatures, noting that they are not cancellable.
- Hazlett 2009: "The mutual assumption of conformity to Grice's maxim of Quality ensures that an utterance of p typically implies that the speaker knows that p". Relies on this in his response to some view (contextualism?). Cited also in Simion 2020 book. (She seems to accept this?)
-- Hazlett, A. (2012) has been cited as opposing cancellability (and perhaps accepting knowledge implicatures?). Perhaps Rysiew 2007 as well, as cited in MBT.
- Aloni and van Rooij "Free Choice Items and Alternatives" 2005 give an account according to which there are knowledge implicatures
- Brown 2012 claims that assertion implicates knowledge but that this is consistent with a variety of epistemic norms for assertion.
- Hawthorne and Dorr:
-- Write about how "if p, q" implicates that the speaker doesn't know p. But in a context where p is clearly known, "if p, q" implies that the speaker *knows* p. (p. 891) Call these 'echoing conditionals'.
-- Moreover it seems to me I could use this for a good example of implicature: "If Bob is here, Alice must be here as well. So Alice must be here". Implicates that Bob is here. Moreover, CANNOT BE CANCELLED! And cannot be knowledge-cancelled. "# But I don't know if Bob is here".
-- Claim again that echoing uses of conditionals have implicatures of knowledge. Does their account depend on this? Possibly: see 896.
-- I *think* they are claiming that echoing conditionals implicate that the speaker knows the antecedent in order to explain why echoing uses of 'if p and might not p ...' are infelicitous, as the speaker cannot know the antecedent in this case (although it would appear that claiming that just the antecedent is implicated, and not knowledge of it, would work just as well; or maybe not -- perhaps knowledge is needed as it is needed in Moorean explanation; see p. 908f).
--- Yes: they do rely on knowledge implicatures, and explain echoing uses in a Moorean fashion. 
-- Explanation of Moorean paradoxality: in asserting the first conjunct, one is in some sense representing oneself as knowing that it is raining, and thereby committing oneself to something inconsistent with what one asserts with the second conjunct.
-- Talk about the 'presumption of transparency', which might be relevant.
-- Holguin on conspiracy conditionals may be relevant here as well?

The centrality of B(pvq) implicature:
- As I understand from Geurts ch. 2, esp p. 41, usually an assertion like "Bob stole some of my apples" will give rise to an implicature that I believe Bob didn't steal *all* of my apples. This is derived by what is called the Standard Recipe.
- The Standard Recipe assumes Competence: where q is implicated that the speaker either believes (or knows) q, or believes not-q. The assertion immediately implicates that the speaker does not believe not-q, whence it follows that the speaker believes q.
- But there is a problem for disjunctions: if I say "either Bob or Alice stole my watch" I imply that I don't know which. But if we follow the Standard Recipe we will get that I don't believe Bob stole my watch, and I don't believe Alice did, which is of course not implied.
- The assumption that there is a *quality* implicature that I believe that either Bob or Alice stole my watch blocks the previous inference. It is in effect in conflict with Competence, and precedes it.
- This is also in Geurts 2009, sec 6.1. Read it to understand this better. Also Schwartz and others cites in these notes. Really drive the point home that Griceans rely on Quality implicatures.
- I should really focus on the 2009, 6.1. part. The quality implicature B(pvq) is assumed in a defense of the Gricean account. I should elaborate on that in a section of my paper to convince that such implicatures are assumed and not so easily dismissed.

More useful stuff:
- Romero M., Han C.. 2004. On negative yes/no questions. Linguistics and Philosophy argue that "isn't p?" (perhaps rhetorical) carry an implicature that the speaker believes p, at least to some degree. (These probably are also not-cancellable)
- "Googled Assertion" claims that one implicates "that he knows more than he actually does" in certain cases.
- Dorst Abominable KK Failures is relevat. Nice Abstract too.
- "Knowledge in the face of conspiracy conditionals" argues that it's infelicitous to assert a condition when you know it's antecedent is true. Maybe KNI can explain it.
- Holguin "Indicative conditionals without iterative epistemology" footnote 10 is useful on KK.
-- The latter perhaps argues for KK as well?
- Perhaps Stokke 2018 (in handbook) is another work on implied contents that I could cite? (Marsili on retweets cites him as such)
- There's also that lying with presuppositions.

Secondary implicature:
- In the scalar implicature lit there is the notion that there are secondary implicatures e.g. if speaker says "some are F", there is the primary implicature that not all are F, and a secondary (epistemic) implicature that the speaker knows/believes that not all are F, i.e., the primary one. (Primary implicature is different though. See Geurts 2009, and link below)
- "Exhaustive interpretation of complex sentences" do give the above primary/secondary distinction (though not in those words). i.e. implicature of ~S, and secondary of K~S. look for "knows" etc.
- Refs: Sauerland 2004, 
- My account easily explains these secondary implicatures. Also cancellation phenomena around them.
- Be careful not to override existing explanations though. See notes here: https://perezcarballo.org/ling753-phil746/files/handout7.pdf
- This is sometimes called the "epsitemic step" e.g. in Faller 2012

\subsection{Cancellation}
- There is all the stuff I have open, and notes from previous doc.
- Sauerland "On Embedded Implicatures" denies cancellation always holds.
- Geurts 2010 (p. 158) hints that implicatures are not always cancellable.
- Guerts 1998 gives a nice example: "she was clever enough to figure out the answer", but cannot be cancelled: "\# she was clever enough to figure out the answers, but she didn't figure out the answer"
- The Magri kind of implicatures ("Some Italians" etc) cannot be cancelled. (See https://jamendia.github.io/files/Mendia2020Snippet.pdf)
- Geurts 2010 has a helpful section on cancellation in chapter 1.
- Atlas 2005 argues against quality implicatures because they are not cancellable, with relation to Moore's paradox. See ref in Faller 2012.
- There's also that introductory book of language that disputes the implicature explanation of Moore's paradox.
- Should write about Irony in this context, and non-cancellability of disbelief. (This is discussed in Haugh 2013)
- Akerman 2015 argues that cancellations does not have to be felicitous.
- Weiner 2006 argues that some implicatures are not cancellable.
- Mayol and Castroviejo 2013 argue that some implicatures are not cancellable, depending on discourse structure.
- H & S 2004 argue that not all particularized implicatures are cancellable.
- Rett 2015 argues against the claim that cancellability is necessary for implicatures, adducing a variety of data.
- Haugh 2013 has a good discussion
- "The Cancellability Test" (Compass 2019)
- Geurts on cancellation:
-- Implicatures can be implicitly cancelled, by context. E.g., in a game where you have to guess where Harry is, I tell you "Harry is either in Antwerp or in Brussels". In this context there is no implicature that I don't know where Harry is.
-- Similarly, I think, belief or knowledge implicatures can be cancelled by context. In some examples it's just irony or metaphor, or overstatement. But other cases as well. As a continuation of the above game we may be guessing what happened to Harry yesterday. I say "Harry is at the bus stop". The implication that I know or believe that Harry is at the bus stop (in the present tense) is cancelled by context.
-- Says better to think of implicatures as *prevented* rather than cancelled. No implicature is literally cancelled he writes.


Think about irony and hyperbole, and sarcasm.
-- "It's snowing, I don't believe it!" is felicitous at least when it's obvious when "I don't believe it" is non-literal. This is a case of hyperbole. The implicature that I sincerely don't believe it is cancelled by my tone, or my asserting that it's snowing.
-- "He is a fine friend. And no, I don't actually believe that". Case of irony.
-- We should start another war. (And no I don't actually believe that)
-- "Just kidding, I don't actually believe that" (Google examples)

Cancellable belief:
- When the implicature is slightly weaker.
-- "Share your opinions about the band. For instance: song 1 is their worst song (and no I don't actually believe that)" [inspired by reddit]
\subsection{Symbols}
- Sauerland uses a squiggly arrow for implicature, and with negation to signal not-implicature. Looks nice. (It seems to be "\\leadsto")
- Geurts uses the same.

\subsection{Implications}
- The constitutive rule theory of assertion: wrong. See "Truth and assertion: rules versus aims", and also Kelp distinguishes the constitutive rule and epistemic norm.
